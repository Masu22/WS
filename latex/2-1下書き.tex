\documentclass{jsarticle}

\begin{document}

\section{文字式の導入}


\subsection*{文字とは?}
数学では「**分からない数**」や「**変わる数**」を使うことが多くなるよ!\\

そんなときに使うのが…\\
👉 **文字**(x, a, b など)!


\subsection*{どうして文字を使うの?}

例1\\
* チョコ1個120円で、ガムは1個80円\\
* チョコを1個、ガムを3個か買いました\\
* 合計で何円になるでしょうか??\\

$120 \times 1 + 80 \times 3 = 120 + 240 =360$\\
なので、合計で360円になる。\\

例2\\
* チョコ1個120円で、ガムは1個80円\\
* チョコを1個、ガムを何個か買いました\\
* 合計で何円になるでしょうか??\\

ガムの個数が分からないので、文字$a$を使ってみる。\\
$a$個ガムを買ったとすると、合計金額は、$120 \times 1 + 80 \times a =120 + 80 \times a$ (円)となる。\\
**文字**を使うと、式が作れるようになるから便利だね!\\

それから、ガムが5個買ったわかったとき、$a$を5にすると、合計金額が計算できる。\\
$120 \times 1 + 80 \times 5 =120 + 400 = 520$円となる。\\

こんな感じで、式を作っておくと、$a$を変えるだけで、すぐに合計金額が計算できる!\\


* 文字を使った式には、書き方のルールがあるので、それを今後勉強していこう!\\


\begin{itemize}
\item 文字式の積のルール
\item 文字式の商のルール、逆数
\item $+-$は省略できない、単位の扱い、文字への代入
\item 文字式の計算例
\item 方程式と基本解法、確かめ算
\end{itemize}
\end{document}